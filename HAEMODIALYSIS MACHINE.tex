\documentclass[12pt]{article}
\usepackage[ a4paper, total={8.1in, 11.5in}]{geometry}
\usepackage{graphicx}
\graphicspath{ {./images/} }
\title{\textbf{HAEMODIALYSIS MACHINE}}
\author{By Harsh Agrawal, 21111021}
\date{}
\begin{document}
\maketitle
\centering{\includegraphics{HAEMODIALYSIS 1}}

\raggedright{\section{\underline{ABOUT DIALYSIS-}}}
The kidneys have an important role in maintaining health. When the person is healthy, the kidneys maintain the body's internal equilibrium of water and minerals (sodium, potassium, chloride, calcium, phosphorus, magnesium, sulphate)
\textbf{Dialysis is a procedure to remove waste products and excess fluid from the blood when the kidneys stop working properly. It often involves diverting blood to a machine to be cleaned.}
Normally, the kidneys filter the blood, removing harmful waste products and excess fluid and turning these into urine to be passed out of the body.

\section{\underline{TYPES OF DIALYSIS:}}
There are Two major types of dialysis-
\begin{itemize}
\item \underline{Haemodialysis-}
Haemodialysis is the most common type of dialysis and the one most people are aware of.
\textbf{Blood passes along the tube and into an external machine known as Haemodialysis Machine that filters it, before it's passed back into the arm along another tube.}
\item \underline{Peritoneal dialysis-}
In peritoneal dialysis, a sterile solution containing glucose \textbf{(called dialysate)} is run through a tube into the peritoneal cavity, the abdominal body cavity around the intestine, where the peritoneal membrane acts as a partially permeable membrane.
No complex machines or blood exchange is required, but it is less effective than Haemodialysis
\end{itemize}
\section{\underline{WORKING PRINCIPLE OF HAEMODIALYSIS:}}
The basic principle of Haemodialysis is \textbf{DIFFUSION}.It involves diffusion of solutes across a semipermeable membrane. Hemodialysis utilizes counter current flow, where the dialysate is flowing in the opposite direction to blood flow in the extracorporeal circuit.
\section{\underline{COMPONENTS OF HAEMODIALYSIS MACHINE:}}
\subsection{\underline{DIALYSATE:}}
Dialysate, also called dialysis fluid, dialysis solution or bath, is a solution of pure water, electrolytes and salts, such as bicarbonate and sodium. The purpose of dialysate is to pull toxins from the blood into the dialysate
\subsection{\underline{Dialysis Machine :}}
A blood pump simply pumps the blood from the body into the machine through specially made tubes.
\subsection{\underline{Syringe:}}
syringe contains a drug called \textbf{Heparin} which prevents the blood from clotting in the tubes.
\subsection{\underline{Dialyzer:}}
The dialyzer is a large canister containing thousands of small fibers through which patient blood is passed. \textbf{The dialyzer is the key part of a dialysis machine where the cleaning of the blood takes place.}
\subsection{\underline{Alarms:}}
order to protect the patient from any errors in functioning. The things that are monitored with alarms include: \textbf{Blood pressure within the machine ,Blood pressure of the patient ,Blood flow ,Temperature ,Dialysate mixture.}
\subsection{\underline{Air Leakage Detector:}}
The detector is settled in the venous blood line and detects as well as in the purpose of avoiding air embolus.
\subsection{\underline{Valves:}}
Several valves with electronic actuation are needed in the machine to allow variable mixing ratios.
\subsection{\underline{Sensors:}}
Dialysis machines require many different types of sensors to monitor various parameters like Blood pressure, O2 saturation, Diasylate mixture.
\section{\underline{WORKING OF HAEMODIALYSIS:}}
\begin{itemize}
\item Two tubes are connected via your hemodialysis access. Blood flows from your body into the machine through one of the tubes. 
\item A pressure monitor and pump work together to keep the flow at the right rate.
\item Your blood enters the \textbf{dialyzer}, where it is filtered.
\item \textbf{Dialysate solution} enters the dialyzer. It draws the waste out of your blood.
\item Used dialysate solution is pumped out of the machine and discarded. 
\item Your blood goes through another pressure monitor and an air trap to make sure it’s safe to go back into your body.
\item Your cleaned blood returns to your body through the \textbf{second tube} attached to your access site.
\end{itemize}
\section {\underline{HAEMODIALYSIS ACCESS:}}
Hemodialysis requires a surgical procedure, to create a connection between your blood vessels. This site is where the blood can flow in and out of your body during the dialysis treatments. This is called the dialysis access.
\textbf{Three types of accesses:}
\begin{itemize}
\item \textbf{Arteriovenous Fistula (AVF):}
A surgically created connection between an artery and a vein, usually in your non dominant arm. This is the preferred type of access because of effectiveness and safety.
\item \textbf{Arteriovenous Graft (AVG):}
If your blood vessels are too small or not available, then the surgeon may use a synthetic material, a graft, to connect your artery to vein.
\item \textbf{Central Venous Catheter (CVC):}
This is usually used if you need emergency hemodialysis. It is also used as a temporary access when AVF or AVG is not yet available
\end{itemize}
\section{\underline{SIDE EFFECTS OF HAEMODIALYSIS:}}
The most common side effects of hemodialysis include \textbf{low blood pressure, access site infection, muscle cramps, itchy skin, and blood clots.}
\section{\underline{SCOPE OF FUTURE INNOVATIONS:}}
The development in haemodialysis can be made as by introducing a \textbf{wearable dialysis} which does not require an on-site visit for dialysis. 
\textbf{Cloud computing} may be used so as to preserve and store the patients’s data.
The dialysis membranes can be improved by \textbf{nanotechnology} so dialysis will be required less frequently.

\end{document}