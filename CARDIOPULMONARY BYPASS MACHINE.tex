\documentclass[12pt]{article}
\usepackage[ a4paper, total={8.1in, 11.5in}]{geometry}
\title{\textbf{CARDIOPULMONARY BYPASS MACHINE(CPB)
			( HEART-LUNG MACHINE)}}
\usepackage{graphicx}
\graphicspath{ {./images/} }
\author{By Harsh Agrawal, 21111021}
\date{}
\begin{document}
\maketitle
\centering{\includegraphics{CPB 1}}

\raggedright{\section{\underline{INTRODUCTION:}}}
The cardiopulmonary bypass machine is also called a heart-lung machine. It takes over for the heart by replacing the heart’s pumping action and by adding oxygen to the blood. 
 \textbf{It is a device that does the work of providing blood (and oxygen) to the body when the heart is stopped for a surgical procedure.} Trained technicians called \textbf{perfusion technologists} (blood flow specialists, also called the “pump team”) operate the heart-lung machine.


\section{\underline{USE OF CARDIOPULMONARY BYPASS MACHINE:}}
The heart-lung machine is used during heart and lung surgeries, so as to keep them \textbf{working and perform the functions of heart and lungs}. To stop the heart without harming the patient, oxygenated blood must continue to circulate through the body during surgery without stopping. The cardiopulmonary bypass pump does the work of the heart, \textbf{pumping blood through the body}, and making sure that the tissues of the body get the oxygen they need.
 The machine also \textbf{adds oxygen to the blood} while taking over the pumping action of the heart, replacing the function of the lungs.

\section{\underline{WORKING PRINCIPLE OF CPB MACHINE:}}
In basics, the cardiopulmonary bypass machine and circuitry are blood reservoirs, blood pumps, and gas exchangers, safety measures along with different ports for gas and drug delivery and exhaust etc. Gas exchange obeys the \underline{law of diffusion} which states that, \textbf{‘the amount of gas diffused across a membrane is proportional to the difference of the partial pressure across the membrane,}
The blood  pumps  works  on  the  basis  of  \underline{Hagen Poiseuille law of flow rate} which states that \textbf{Flow is inversely proportional to the viscosity of the fluid.}

\section{\underline{COMPONENTS OF CPB MACHINE:}}
\subsection{\underline{PUMP:}}
There are two types of pumps that can be used in CPB machine-
\begin{itemize}
\item Roller pump- 
Roller pump includes two rollers positioned on a rotating arm, which compress a length of tubing to produce forward flow. This action gently propels the blood through the tubing.

\item Centrigugal Pump-
 Centrifugal pump consists of impellers/stacked cones within housing. When rotated rapidly, negative pressure is created at one inlet, and positive pressure at the other, thus propelling the blood forward. By altering the speed of revolution (RPM) of the pump head, blood flow is produced by centrifugal force. 
\end{itemize}
\subsection{\underline{CANNULAE :}}
Cannulae connect the patient to the circuit and hence to the CPB machine. They are made of polyvinylchloride (PVC) and are wire reinforced to prevent obstruction.
There are two types of cannula used in the heart during CPB-
\begin{itemize}
\item Single-stage cannulae - used during most open-heart surgeries, where two cannulae are inserted into the superior and inferior vena cava and joined by a Y-piece.
\item Dual-stage cannulae - used for most closed-heart procedures, where a single cannula is inserted into the right atrium. Drainage occurs through gravity.
\end{itemize}
\textbf {Venous cannulation-} Intravenous (IV) cannulation is a technique in which a cannula is placed inside a vein to provide venous access

\subsection{\underline{OXYGENATOR:}}
The oxygenator is designed to transfer oxygen to \textbf{infused blood and remove carbon dioxide} from the venous blood.Earlier, Bubble oxygenators were used, but now \textbf{Membrane oxygenators} are used. Membrane oxygenators consist of hollow microporous polypropylene fibres.  Blood flows outside the fibre while gases pass inside the fibre, thus separating the blood and gas phases.
\subsection{\underline{Heat Exchanger:}}
A heat exchange system used in \textbf{extracorporeal circulation} to warm or cool the blood or perfusion fluid flowing through the device
\newline \textbf{Extraporeal Circulation-}- A procedure in which blood is taken from a patient's circulation to have a process applied to it before it is returned to the circulation.
\subsection{\underline{TUBING:}}
These are generally made of PVC, due to PVC's durability and acceptable \textbf{haemolysis rate (loss of blood rate)}.
\subsection{\underline{RESERVOIR:}}
They collect the blood drained from the heart. \textbf{Open reservoirs} are more commonly used. They allow passive removal of entrained venous air along with the option of applying vacuum to assist drainage. When they are used, a safe level of blood in the reservoir is maintained to avoid air entry into the arterial circuit. \textbf{Closed reservoirs} have a limited volume capacity, but offer a smaller area of blood contact with artificial surfaces.
\subsection{\underline{GAS LINE AND BLENDER :}}
Delivers fresh gas to the oxygenator in a controlled mixture.
\subsection{\underline{CARDIOPLOEGIA SOLUTION AND CANNULA:}}
The blood after being processed from oxygenator is then returned to the patient via an arterial cannula positioned in the ascending aorta or other major artery.
Additional CPB circuit pumps or other components deliver cardioplegia solution to produce \textbf{cardiac electromechanical silence(i.e. to stop the heart from beating)}, which reduces the oxygen consumption, decompress the heart via a vent, and remove fluid (ultrafiltration).
\subsection {\underline{ACT MACHINES:}}
Automated machines used to measure the \textbf{Activated clotting time}, which is the effectiveness of heparin, to prevent the clotting of blood.

\section{\underline{WORKING OF A CPB MACHINE:}}
A special tubing is attached  to a large blood vessel \textbf{that allows oxygen-depleted blood to leave the body} and travel to the bypass machine. There, the machine oxygenates the blood and returns it to the body through the second set of tubing, also attached to the body. The constant pumping of the machine pushes the oxygenated blood through the body, much like the heart does.
The tubes must be placed in a blood vessel large enough to accommodate the tubing and the pressure of the pump. The two tubes ensure that blood leaves the body before reaching the heart and returns to the body after the heart, giving the surgeon a still and mostly bloodless area to work.

\section{\underline{COMPLICATIONS OF CPB MACHINE:}}
Complications include oxygenator failure, pump malfunction, clotting in the circuit, tubing rupture, gas supply failure and electrical failure
CPB causes decrease in Platelets.
Clotting of blood due to mismanagement of ACT and Heparin may lead to serious complications.
\section{\underline{SCOPE OF FUTURE INNOVATIONS IN CPB:}}
The CPB machines may be further improved by using technology such as \textbf{nanotechnology to developed semipermeable membranes} for better filtration of blood. ACT machines may be further developed using \textbf{Artificial Intelligence To expect accurate clotting time}. \textbf{Machine Learning} Can be used to monitor and alert about the vital parameters or the effectiveness of Heparin.
\end{document}