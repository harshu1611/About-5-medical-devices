\documentclass[12pt]{article}
\usepackage[ a4paper, total={8.1in, 11.5in}]{geometry}
\usepackage{graphicx}
\graphicspath{ {./images/} }
\title{\textbf{ELECTROCARDIOGRAM(ECG)}}
\author{By Harsh Agrawal, 21111021}
\date{}
\begin{document}
\maketitle
\centering{\includegraphics{ECG 1}}

\raggedright{\section{\underline{Introduction To ECG:}}}
Electrocardiogram or ECG is a  process of recording of the heart's electrical activity. It is an electrogram of the heart which is a graph of voltage versus time of the electrical activity of the heart using electrodes placed on the skin.
An electrocardiogram (ECG) is a simple test that can be used to check your \textbf{heart's rhythm and electrical activity}.
Sensors attached to the skin are used to detect the electrical signals produced by your heart each time it beats.
It's a common and painless test used to quickly detect heart problems and monitor your heart's health.
The device used is known as ECG Machine.
\section{\underline{Diagnosis By ECG:}}
ECG machine is helpful in diagnosing-
\begin{itemize}
\item Abnormal heart rhythm (arrhythmias)
\item Blockage in any part of the heart.
\item Heart attack.
\end{itemize}
\section{\underline{WORKING PRINCIPLE OF AN ECG MACHINE:}}
The basic principle of the ECG is that \textbf{stimulation of a muscle alters the electrical potential of the muscle fibres}. 
Cardiac cells, unlike other cells, have a property known as \textbf{automaticity}, which is the capacity to spontaneously initiate impulses.
In simple words, we can say that an ECG works on the 
\textbf{responsiveness of the muscles after passing an electrical signal.}
\section{\underline{COMPONENTS OF ECG MACHINE:}}
\subsection{\underline{Electrodes:}}
Electrodes consist of two types, \textbf{the bipolar and unipolar}. The bipolar electrodes can be placed on both the wrists and the legs to measure the voltage differential between the two.
Unipolar electrodes, measure the voltage difference or the electrical signal between a special reference electrode and actual body surface while being placed on both arms and legs.The ECG comprises 12 leads.
\subsection{\underline{Amplifiers:}}
\textbf{The amplifier reads the electrical signal in the body and prepares it for the output device.} When the electrode's signal reaches the amplifier it is first sent to the buffer, the first section of the amplifier. When it reaches the buffer, the signal is stabilized and then translated.
\subsection{\underline{Connecting Wires:}}
\textbf{The connecting wires transmit the signal read from the electrodes and send it to the amplifier.} These wires connect directly to the electrodes; the signal is sent through them and connected to the amplifier.
\subsection{\underline{Output:}}
The output is a device on the ECG where the electrical activity of the body is processed and then recorded onto graph paper.
\section{\underline{WORKING OF ECG:}}
\begin{itemize}
\item For an electrocardiogram, a person is made to lie in the resting position and electrodes are placed on arms, legs and at six places on the chest over the area of the heart. The electrodes are attached to the person’s skin with the help of a special jelly. A total of 10 electrodes (points of contact with the body) are used to perform an ECG.
\item The electrode picks up the current and transmit them to an amplifier inside the electrocardiograph. Then electrocardiograph amplifies the current and records them on a paper as a wavy line.
\item In an electrocardiograph, a sensitive lever traces the changes in current on a moving sheet of paper.
\item Modern ECGs may also be connected to a screen which displays the current and the waves.
\item A normal ECG makes a specific pattern of three recognizable waves in a cardiac cycle. These wave are- \textbf{P wave, QRS wave and T-wave, P-R interval, S-T segment.}
\end{itemize}
\section {\underline{LIMITATIONS OF ECG:}}
The ECG reveals the heart rate and rhythm only during the few seconds it takes to record the tracing.
Sometimes the electrodes placed in the body may hurt the skin.
Sometimes, The ECG is often normal or nearly normal with many types of heart disease.
\section{\underline{SCOPE OF FUTURE INNOVATIONS:}}
ECG machines may be further developed using a better electrode technology for faster and accurate diagnosis.
generating a \textbf{3D structure or report of heart} using ECG may be helpful.
ECG machines are now also being developed to check the harmful effects of drugs on heart.
Using cloud computing and generating reports on mobile phones may be developed, and usage of less number of electrodes in the body.
A \textbf{non-touch method} of ECG may be developed to prevent skin problems and better report.

\end{document}