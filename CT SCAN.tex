\documentclass[12pt]{article}
\usepackage[ a4paper, total={8.1in, 11.5in}]{geometry}
\usepackage{graphicx}
\graphicspath{ {./images/} }
\title {\textbf{CT SCAN MACHINE(COMPUTED TOMOGRAPHY)}}
\author{By Harsh Agrawal,21111021}
\date{}
\begin{document}
\maketitle 
\centering{\includegraphics{CT SCAN 1}}
\section{ \raggedright{{\underline{INTRODUCTION ABOUT CT SCAN}}}}
\raggedright{A \textbf{computerized tomography (CT)} scan combines a series of X-ray images taken from different angles 
around your body and uses computer processing to create cross-sectional images (slices) of the bones,
blood vessels and soft tissues inside your body. CT scan images provide more-detailed information than plain X-rays do.}
\subsection{What is Tomography?}
Tomography is imaging by sections or sectioning through the use of any kind of penetrating wave. The method is used in radiology, archaeology, biology, atmospheric science, geophysics, oceanography, plasma physics, materials science, astrophysics, quantum information, and other areas of science.
The basic meaning of tomography is \textbf{producing images by passing any ray or wave.}
\section{\underline{WORKING OF A CT SCAN MACHINE}}
The main concept behind working of a CT machine is same as the other imaging devices, which uses X-ray beams for the scan. But Unlike x-ray machine which uses a fixed x-ray tube, the CT scan machine uses a \textbf{motorized x-ray tube}  that rotates around the circular opening of a donut-shaped structure called a \underline{gantry}. During a CT scan, the patient lies on a bed that slowly moves through the gantry while the \textbf{x-ray tube rotates around the patient, shooting narrow beams of x-rays through the body.} One more difference from x-ray machines is that instead of films, CT scanners use special digital \textbf{x-ray detectors}, which are located directly opposite the x-ray source. As the x-rays leave the patient, they are picked up by the detectors and transmitted to a computer.
\newline Each time the x-ray source completes one full rotation, the CT computer uses sophisticated mathematical techniques to construct a \textbf{2D image slice} of the patient. The thickness of the tissue represented in each image slice can vary depending on the CT machine used
Image slices can either be displayed individually or stacked together by the computer to generate a 3D image of the patient that shows the skeleton, organs, and tissues as well as any abnormalities the physician is trying to identify.
\section{\underline{COMPONENTS OF A CT SCANNER}}
\subsection{\underline {GANTRY}}
The gantry is the donut like or ring shaped part of the CT scanner. It houses many of the components necessary to produce and detect X-rays. These components are mounted on a rotating scan frame.
\subsection{\underline {Slip Rings}}
Newer CT systems electromechanical devices called slip rings . Slip rings use a brush like apparatus to provide continuous electrical power and electronic communication across a rotating surface.
\subsection{\underline {Generator}}
High frequency generator is currently used in CT scanners. The generator are designed to be small enough so that it can be located within the gantry. Generators produce high voltage and transmit it to the xray tube. The power capacity of the generator is listed in kilowatts (kW).
\subsection{\underline {Cooling Systems}}
Cooling mechanisms are included in the gantry. Cooling mechanisms are important because many components can be affected by temperature fluctuations
\subsection{\underline {X-ray Source – CT X-ray tube}}
X-ray tubes produce the xray photons that create the CT image. It is the main component of ct scanner as it works on the principle of x-ray and has a motorized x-ray tube.
\subsection{\underline {Filtration}}
Compensating filters are used to shape the X-ray beam. They reduce the radiation dose to the patient and help to minimize image artifact. As we know that X- rays maybe harmful for the body, Filters are necessary to minimize the harmful effects.
\subsection{\underline {Collimation}}
Collimation restrict the X-ray beam to a specific area, as a result it helps reduce scatter radiation. This scatter radiation reduces image quality and increase the radiation dose to the patient.
\subsection{\underline {SOFTWARES USED IN CT SCAN-}}
The main software used in CT scan machines are called medical imaging softwares.
These imaging softwares process the x-ray data collected by detector from ct machine and converts them to digital images. Some medical imaging softwares are Planmed, mediCAD,inSimo, etc.
\newline
\newline \textbf{\raggedright{\underline{{SCOPE OF INNOVATIONS IN CT SCAN-}}}}
\newline Innovations in the field of medical imaging devices can be made as to reduce the harmful effects of rays emitting, improving the quality of image producing, using advanced detectors to easily detect the rays, a compact CT scan can be made using the technology of \textbf{cloud computing and wireless technologies.}


\end{document}